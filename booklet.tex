\documentclass[a4paper,10pt]{book}
\usepackage[utf8]{inputenc}
\usepackage[width=17cm,height=27cm,includeheadfoot]{geometry}
\usepackage[pdftex,colorlinks=true]{hyperref}
\usepackage{fancyhdr,lastpage}
\usepackage{tabularx}
\usepackage[table]{xcolor}


%opening
\title{SocInfo2012}
\author{}

\begin{document}
\pagestyle{fancy}
\renewcommand{\headrulewidth}{1pt}
\renewcommand{\footrulewidth}{1pt}
\renewcommand{\headheight}{14pt}
\fancyhead[L]{SocInfo2012}
\fancyhead[R]{The 4th International Conference on Social Informatics}
\fancypagestyle{plain}{
	\renewcommand{\headrulewidth}{1pt}
	\renewcommand{\footrulewidth}{1pt}
}

\maketitle

\tableofcontents



\newcommand{\ncell}[2][X]{%
\begin{tabularx}{0.71\linewidth}[#1]{@{}X@{}}#2\end{tabularx}%
}
  
\chapter{Programme Overview}
NOTE: We should indicate the rooms in the tables
\begin{center}
\bgroup
\def\arraystretch{1.5}
\begin{tabularx}{.9\linewidth}{|c|X|X|}
\hline
\multicolumn{3}{|c|}{\textbf{Wednesday, December 5}}\\
\hline
08:00 $\rightarrow$ 18:00 & \multicolumn{2}{c|}{Registration}\\
\hline
08:45 $\rightarrow$ 09:00 & \multicolumn{2}{c|}{Conference opening}\\
\hline
09:00 $\rightarrow$ 10:00 & \multicolumn{2}{c|}{\cellcolor{green!20}
\ncell{Keynote by Andreas Ernst:\\About the Why? and the How? of psychologically
plausible agents}}\\
\hline
10:00 $\rightarrow$ 10:30 & \multicolumn{2}{c|}{Break}\\
\hline
10:30 $\rightarrow$ 12:30 & \multicolumn{2}{c|}{\cellcolor{blue!20}Session
1: Social Graph, Social Influence and Viral Marketing}\\
\hline
12:30 $\rightarrow$ 14:00 & \multicolumn{2}{c|}{Lunch break}\\
\hline
14:00 $\rightarrow$ 16:00 & \multicolumn{2}{c|}{\cellcolor{blue!20}Session 2:
Recommendation and Crowd Computing }\\
\hline
16:00 $\rightarrow$ 16:30 & \multicolumn{2}{c|}{Break}\\
\hline
16:30 $\rightarrow$ 18:00 & \cellcolor{red!20} Tutorial 1: Supporting
sociological theories with social media data mining  &
\cellcolor{red!20} Tutorial 2: Online Social Experiments with nodeGame \\
\hline
\end{tabularx}
\egroup
\end{center}

\begin{center}
\bgroup
\def\arraystretch{1.5}
\begin{tabularx}{.9\linewidth}{|c|X|}
\hline
\multicolumn{2}{|c|}{\textbf{Thursday, December 6}}\\
\hline
09:00 $\rightarrow$ 10:00 & \multicolumn{1}{c|}{\cellcolor{green!20}
\ncell{Keynote by Bernardo Huberman:\\Big Data and the Attention Economy}}\\
\hline
10:00 $\rightarrow$ 10:30 & \multicolumn{1}{c|}{Break}\\
\hline
10:30 $\rightarrow$ 12:30 & \multicolumn{1}{c|}{\cellcolor{blue!20}Session
3: Sentiment Analysis and Trust}\\
\hline
12:30 $\rightarrow$ 14:00 & \multicolumn{1}{c|}{\cellcolor{red!20}Buffer lunch
with poster/demo session}\\
\hline
14:00 $\rightarrow$ 16:00 & \multicolumn{1}{c|}{\cellcolor{blue!20}Session 4:
Social Tagging and Discovery}\\
\hline
16:00 $\rightarrow$ 16:30 & \multicolumn{1}{c|}{Break}\\
\hline
16:30 $\rightarrow$ 18:00 & \multicolumn{1}{c|}{\cellcolor{red!20}
\ncell{Tutorial 3: Human activity and
mobility patterns: measurements, models,\\ and implications }} \\
\hline
\end{tabularx}
\egroup
\end{center}

\begin{center}
\bgroup
\def\arraystretch{1.5}
\begin{tabularx}{.9\linewidth}{|c|X|}
\hline
\multicolumn{2}{|c|}{\textbf{Friday, December 7}}\\
\hline
09:00 $\rightarrow$ 10:00 & \multicolumn{1}{c|}{\cellcolor{green!20}
\ncell{Keynote by Dirk Helbing:\\ From Computational Social Science to
Socio-Inspired Technology to Artificial Societies}}\\
\hline
10:00 $\rightarrow$ 10:30 & \multicolumn{1}{c|}{Break}\\
\hline
10:30 $\rightarrow$ 12:30 & \multicolumn{1}{c|}{\cellcolor{blue!20}Session 5:
Community Detection and Evolution}\\
\hline
12:30 $\rightarrow$ 14:00 & \multicolumn{1}{c|}{Lunch break}\\
\hline
14:00 $\rightarrow$ 16:00 & \multicolumn{1}{c|}{\cellcolor{blue!20}Session 6:
Social Informatics and Applications }\\
\hline
16:00 $\rightarrow$ 16:30 & \multicolumn{1}{c|}{Break}\\
\hline
16:30 $\rightarrow$ 18:00 & \multicolumn{1}{c|}{\cellcolor{red!20}
Plenary Panel} \\
\hline
\end{tabularx}
\egroup
\end{center}
\chapter{Abstracts per session}
\section{Session 1: Social Graph, Social Influence and Viral Marketing
}
\begin{itemize}
\item \textbf{Connecting with Active People Matters: The Influence of an Online
Community on Physical Activity Behavior}

\textit{Maartje Groenewegen, Dimo Stoyanov, Dirk Deichmann and Aart van
Halteren}

This paper discusses the impact of online social networks as means to motivate
people to become more physically active. Based on a data set from 4333
participants we show that the activity level of people that participated in the
online community (for 14 weeks) is significantly higher compared to people that
choose not to become a member of that community. Detailed analyses show that the
number of contacts in the online community does not have a significant effect on
the physical activity level while network density even has a significant,
negative effect. On the other hand, the activity level of a participant is
higher when his or her friends also have a high average activity level. This
effect is even higher when a participant's amount of friends
increases.Theoretical and managerial implications concerning the impact of
online social networks on offline behavior are discussed. 

\item \textbf{The Multidimensional Study of Viral Campaigns as Branching
Processes}

\textit{Jaroslaw Jankowski, Radoslaw Michalski and Przemyslaw Kazienko}

Viral campaigns on the Internet may follow variety of models, depending on the
content, incentives, personal attitudes of sender and recipient to the content
and variety of other factors. Due to the fact that the knowledge of the campaign
specifics is essential for the campaign managers, researchers are constantly
evaluating models and real-world data. The goal of this article is to present
the new knowledge obtained from studying two viral campaigns that took place in
a virtual world which followed the branching process. The results show that it
is possible to reduce the time needed to estimate the model parameters of the
campaign and, moreover, some important aspects of time-generations relationship
are presented.

\item \textbf{A Model to Represent Human Social Relationships in Social Network
Graphs}

\textit{Marco Conti, Andrea Passarella and Fabio Pezzoni}

Human social relationships are a key component of emerging complex techno-social
systems such as socially-centric platforms based on the interactions between
humans and ICT technologies. Therefore, the models of human social relationships
are fundamental to characterise these systems and study the performance of
socially-centric platforms depending on the social context where they operate.
The goal of this paper is presenting a generative model for building synthetic
human social network graphs where the properties of social relationships are
accurately reproduced. The model goes well beyond a binary approach, whereby
edges between nodes, if existing, are all of the same type. It sets the
properties of each social link, by incorporating fundamental results from the
anthropology literature. The synthetic networks it generates accurately
reproduce both the macroscopic structure (e.g., its diameter and clustering
coefficient), and the microscopic structure (e.g., the properties of the tie
strength of individual social links) of human social networks. We compare
generated networks with a large-scale social network data set, validating that
the model is able to produce graphs with the same structural properties of
human-social-network graphs. Moreover, we characterise the impact of the model
parameters on the synthetic graph properties.


\item \textbf{Interpolating between Random Walks and Shortest Paths: a Path
Functional Approach}

\textit{Francois Bavaud and Guillaume Guex}

General models of network navigation must contain a deterministic or drift
component, encouraging the agent to follow routes of least cost, as well as a
random of diffusive component, enabling free wandering. This paper proposes a
thermodynamic formalism involving two path functionals, namely an energy
functional governing the drift and an entropy functional governing the
diffusion.\\
A freely adjustable parameter, the temperature, arbitrates between the
conflicting objectives of minimising travel costs and maximising spatial
exploration. The theory is illustrated on various graphs and various
temperatures. The resulting optimal paths, together with presumably new
associated edges and nodes centrality indices, are analytically and numerically
investigated. 

\item \textbf{Studying Paths of Participation in Viral Diffusion Process within
Virtual Chat Environment}

\textit{Jaroslaw Jankowski, Sylwia Ciuberek, Anita Zbieg and Radoslaw Michalski}

Authors propose a conceptual model of participation in viral diffusion process
composed of four stages: awareness, infection, engagement and action. To verify
the model it has been applied and studied in the virtual social chat environment
settings. The study investigates the behavioural paths of actions that reflect
the stages of participation in the diffusion and presents shortcuts, that lead
to the final action – the attendance in a virtual event. The results show that
the participation in each stage of the process increases the probability of
reaching the final action. Nevertheless, the majority of users involved in the
virtual event did not go through each stage of the process but followed the
shortcuts. That suggests that the viral diffusion process is not necessarily a
linear sequence of human actions but rather a dynamic system.


\item \textbf{How Influential are You: Detecting Influential Bloggers in a
Blogging Community}

\textit{Imrul Kayes}

The emergence of advanced web 2.0 technologies has created a new horizon for
previously known information consumer, enabled them to produce information on
the web through novel and innovative inter- active applications such as blogs. A
Blog is a virtual media on the web where users commonly referred as bloggers
publish their context aware information. Bloggers write blog posts, express a
preference through likes and dislikes, voice their opinions, participate debate,
report news, and form virtual communities of similar interest groups on the
Blogosphere. Bloggers interactions on blogosphere mimic, to some extent, real
world interaction. Inspired by the high impact of the influential individuals in
a real world community, we study a contemporary problem of identifying
influential bloggers at a blogging platform. We model bloggers influence based
upon a wide range of centrality measurements and quantify influence strength. We
conduct experiments with data from a real world blogging platform, discover
multi-facets of the problem of identifying influential bloggers, and discuss
plausible methods. Our study reveals that some bloggers span MEGA influence on
fellow bloggers and there is some degree of correlation between the methods that
are used along the uncovering process.


\item \textbf{Dark Retweets: Investigating Non-Conventional Retweeting Patterns}

\textit{Norhidayah Azman, David Millard and Mark Weal}

Retweets are an important mechanism for recognising propagation of information
on the Twitter social media platform. However, many retweets do not use the
official retweet mechanism, or even community established conventions, and these
"dark retweets" are not accounted for in many existing analysis. In this paper,
a comprehensive matrix of tweet propagation is presented to show the different
nuances of retweeting, based on seven characteristics: whether it is
proprietary, the mechanism used, whether it is directed to followers or
non-followers, whether it mentions other users, if it is explicitly propagating
another tweet, if it links to an original tweet, and what is the audience it is
pushed to. Based on this matrix and two assumptions of retweetability, the
degrees of a retweet's "darkness" can be determined. This matrix was evaluated
over 2.3 million tweets and it was found that dark retweets amounted to 12.86\%
(for search results less than 1500 tweets per URL) and 24.7\% (for search
results including more than 1500 tweets per URL) respectively. By extrapolating
these results with those found in existing studies, potentially thousands of
retweets may be hidden from existing studies on retweets.


\end{itemize}
\end{document}
